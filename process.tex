\documentclass{article}
\usepackage{graphicx}

\begin{document}

\title{Software Process}

\maketitle

\begin{abstract}
	A sequence of steps to carry out a software developement effort in order
	to effectively and predictably deliver a solution to proposed
	requirements is detail. The described process is a lightweight
	iterative process. It can be succinctly described as a step of initial
	requirement definition followed by a series of sprints emulating small
	waterfall lifecycles where deficiencies in the process application,
	requirements, and design are noted and repaired in between sprints. Once
	the requirements have been met, a final deliver step is taken. The
	entire process is meant to be followed from the beginning to the end of
	a major version. Artifacts for different versions should be labelled
	by the version and contain a complete set of process artifacts.
\end{abstract}

\section{Process}

\begin{enumerate}
	\item Requirements analysis \begin{itemize}
		\item Activities: \begin{enumerate}
			\item Elicit requirements: Collect requirements for
				entire project.
			\item Document requirements: More specifically design
				and document requirements. Add as issues on
				Github under version milestone.
			\item Specify delivery procedure: Document in what way
				the project will be delivered.
			\item High level design: Specify the high-level design
				of the solution.
			\item Determine success model: List criteria that when
				met says the project is complete.
			\item Estimate success time: Attempt to determine a time
				until success model is met.
			\item Prototype to gain more insight into other
				activities.
		\end{enumerate}
			\item Artifacts: \begin{itemize}
				\item doc/learned.pdf: Update learned
					information.
				\item doc/requirements.pdf: Record
					requirements in terms of use cases, user
					stories, or other medium along with
					deliver procedure and success model.
				\item doc/design.pdf: Specify solution design,
					including visual design, architecture,
					and collaboration along with tools and
					languages.
				\item README.md: Short description of the
					projectand what requirements it intends
					to meet.
				\item LICENSE: Use conditions for the project.
				\item Prototype: Prototyped code.
			\end{itemize}
		\end{itemize}
	\item Waterfall Iterations (Loop while success model not met)
		\begin{itemize}
			\item Activities: \begin{enumerate}
				\item Requirements analysis: Specify and refine
					requirements to be completed during this
					iteration. Update Github issues with
					deadline and information. Also include
					future requirements.
				\item Design: Document a potential solution for
					the requirements. Choose languages and
					build tools. Report validation and
					verification strategy. Include impact of
					design on future requirements.
				\item Code: Articulate solution into code.
				\item Validate and verify: Make sure
					requirements were met and requirements
					were met correctly using strategy
					specified in design. Review code. Split
					into patterns if useful. Mark Github
					issues as completed.
			\end{enumerate}
			\item Artifacts: \begin{itemize}
				\item doc/learned.pdf: Update learned
					information.
				\item doc/requirements.pdf: Include more
					specific, formed, or new requirements.
				\item doc/design.pdf: Include more concrete
					information about design.
				\item README.md: Discuss new relevant
					information.
			\end{itemize}
		\end{itemize}
	\item Deliver \begin{itemize}
		\item Activites: \begin{enumerate}
			\item Validate and verify: Ensure the correct problem
				has been solved and has been solved correctly.
			\item Execute delivery procedure: Deliver the project to
				the intended audience with the intended
				strategy.
		\end{enumerate}
		\item Artifacts: \begin{itemize}
			\item doc/learned.pdf: Update learned
				information.
			\item doc/requirements.pdf: Finalized requirements
				document.
			\item doc/design.pdf: Finalized design document.
			\item README.md: Finalized README.md.
		\end{itemize}
	\end{itemize}
\end{enumerate}

\section{Tools}

\begin{itemize}
	\item Build tools: \begin{itemize}
		\item Make
		\item npm
	\end{itemize}
	\item Version control: \begin{itemize}
		\item Git
		\item Github
	\end{itemize}
	\item Editors: \begin{itemize}
		\item Vim: Small or independent projects.
		\item JetBrains: Larger or team based projects. Choose
			based on language.
	\end{itemize}
\end{itemize}

\section{Versions}

Semantic versioning should always be used. Each major and minor version should
be tagged in version control. Semantic versions are of the form x.y.z where x is
the major version, y the minor version, and z the patch version. Major versions
are for breaking changes. Minor versions are when backwards compatabile
functionality is added. Patch versionsare for backwards compatible bug fixes.

\section{Languages}

\begin{itemize}
	\item General Purpose: Go
	\item Application Development: Java
	\item Web Development: Javascript, HTML, CSS, Javascript
		\begin{itemize}
		\item Libraries: \begin{itemize}
			\item SASS
			\item Angular2
			\item Typescript
		\end{itemize}
		\end{itemize}
	\item Prototyping and Glue: Python
	\item High Performance: C / C++
\end{itemize}

\end{document}
